\documentclass[twocolumn]{article}
%\usepackage{multicol}

%opening
\title{User Error}
\author{}
\date{}

\begin{document}

\maketitle

\section*{Authors}

\emph{User Error} was created by Jeremy Vasta, Sam Bloomberg, Liam Middlebrook, Alex Harper, Oliver Barnum, and Alex Hoshino.

\section*{Premise}

Welcome to your job as a manager for an outsourced IT firm! You are going to be competing against other firms for the satisfaction of our company. Careful, though, if things get too chaotic the entire company might go under!

\section*{Materials}

\begin{itemize}
	\item Asset cards
	\item Issue cards
	\item Contract cards
	\item Audit cards
	\item Stability + turn sheet
	\item Stability + turn markers
	\item 4 $\times$ satisfaction sheets
	\item 4 $\times$ satisfaction markers
	\item 6-sided die
	\item First player marker
	\item ? $\times$ budget counters
\end{itemize}

\pagebreak

\section*{Setup}

Start by separating out the four decks of cards and shuffling them. Place each deck on the table. Place the stability/turn sheet in the center where all players can see it, and place the turn and stability markers at the beginning of their respective tracks.

Each player receives a satisfaction sheet, and places a satisfaction marker at the 0 position on their sheet. Each player also receives 3 budget.

The first player marker starts with whomever last called IT or tech support.

\subsection*{Initial Acquisition}

Deal a number of asset cards face up on the table equal to 2 $\times$ the number of players. Players take turns starting with the first player and continuing in a clockwise direction, buying a single asset at a time. This continues until no one wants to or is able to buy any more assets. Players are not forced to buy assets, though it is in their best interest to do so.

Once initial acquisition is completed, deal enough asset cards face up until there are at least as many as to match the number of players.

\section*{Winning and Losing}

If at any time the stability marker is at or past the last space on the stability track, all players lose regardless of their satisfaction.

After the tenth turn, the player with the highest satisfaction wins the game.

\pagebreak

\section*{Phases}

Once initial acquisition is completed, play continues with the following phases for the rest of the game.

\subsection*{Dispatch}

Players are dealt a number of corresponding to the current row the stability marker is at into their backlog. Players may now assign at most one issue per technician.

\subsection*{Troubleshooting}

Players take turns (starting with the first player, and proceeding in a clockwise direction) either
resolving or referring a single issue. This phase ends once actions have been taken for all assigned issues.

\subsubsection*{Resolving Issues}

To resolve an issue, announce which issue/technician you are attempting to resolve. Roll the D6 and add the technician's skill to the value on the die. If the technician gives a bonus to the type of issue you are trying to resolve, add that as well. If the total is equal to or greater than the difficulty of the issue, the issue is resolved, discarded, and you may move your satisfaction marker up by one.
If you fail to resolve the issue, it is returned to your backlog.

\subsubsection*{Referring Issues}

If an issue has a difficulty that exceeds your technician's skill by at least 4, you may put the issue into any other player's backlog. That issue may not be further resolved or referred until the next troubleshooting phase.


\subsection*{Negotiation}

Two contract cards are dealt face-up on the table.

\subsection*{Acquisition}

\subsection*{Evaluation}

\end{document}
