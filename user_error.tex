\documentclass[twocolumn]{article}

%opening
\title{User Error}
\author{}
\date{}

\begin{document}

\maketitle

\section*{Authors}

\emph{User Error} was created by Jeremy Vasta, Sam Bloomberg, Liam Middlebrook, Alex Harper, Oliver Barnum, and Alex Hoshino.

\section*{Premise}

Welcome to your job as a manager for an outsourced IT firm! You are going to be competing against other firms for the satisfaction of our company. Careful, though, if things get too chaotic the entire company might go under!

\section*{Materials}

\begin{itemize}
	\item Asset cards
	\item Issue cards
	\item Contract cards
	\item Audit cards
	\item Stability + turn sheet
	\item Stability + turn markers
	\item 4 $\times$ satisfaction sheets
	\item 4 $\times$ satisfaction markers
	\item 6-sided die
	\item First player marker
	\item ? $\times$ budget counters
\end{itemize}

\pagebreak

\section*{Setup}

Start by separating out the four decks of cards and shuffling them. Place each deck on the table. Place the stability/turn sheet in the center where all players can see it, and place the turn and stability markers at the beginning of their respective tracks.

Each player receives a satisfaction sheet, and places a satisfaction marker at the 0 position on their sheet. Each player also receives 3 budget.

The first player marker starts with whomever last called IT or tech support.

\subsection*{Initial Acquisition}

Deal a number of asset cards face up on the table equal to 2 $\times$ the number of players. Players take turns starting with the first player and continuing in a clockwise direction, buying a single asset at a time. This continues until no one wants to or is able to buy any more assets. Players are not forced to buy assets, though it is in their best interest to do so.

Once initial acquisition is completed, all remaining asset cards are shuffled back into the asset deck.

\section*{Winning and Losing}

If at any time the stability marker is at or past the last space on the stability track, all players lose regardless of their satisfaction.

After the tenth turn, the player with the highest satisfaction wins the game.

\pagebreak

\section*{Phases}

Once initial acquisition is completed, play continues with the following phases for the rest of the game.

\subsection*{Dispatch}

Players are dealt a number of corresponding to the current row the stability marker is at into their backlog. Players may now assign at most one issue per technician.

\subsection*{Troubleshooting}

Players take turns (starting with the first player, and proceeding in a clockwise direction) either
resolving or referring a single issue. This phase ends once actions have been taken for all assigned issues.

\subsubsection*{Resolving Issues}

To resolve an issue, announce which issue/technician you are attempting to resolve. Roll the D6 and add the technician's skill to the value on the die. If the technician gives a bonus to the type of issue you are trying to resolve, add that as well. If the total is equal to or greater than the difficulty of the issue, the issue is resolved, discarded, and you may move your satisfaction marker up by one.
If you fail to resolve the issue, it is returned to your backlog.

\subsubsection*{Referring Issues}

If an issue has a difficulty that exceeds your technician's skill by at least 4, you may put the issue into any other player's backlog. That issue may not be further resolved or referred until the next troubleshooting phase.


\subsection*{Negotiation}

Two contract cards are dealt face-up on the table. Players now take turns bidding on the first dealt contract, starting with whomever has the lowest satisfaction (ties are broken by whomever is closest to the first player) and proceeding in a clockwise direction. Bidding may start at any number equal to or less than the budget value on the contract. Once there are no players willing to take the contract for less budget, the last player to bid wins the contract and earns whatever amount of budget they agreed to. The contract comes into effect immediately.
This repeats again for the second contract that was dealt except that the player who just won the first contract may not bid on the second.

\subsubsection*{Audits}

If either contract that was dealt is an \emph{Audit} card, then an audit occurs. All abandoned contracts must be discarded and players who had those contracts lose any budget associated with them, firing any assets to make up for that loss of budget. Any assets that have \emph{audit risk} written on them are automatically fired as well, with the owning player gaining whatever budget those assets were worth.

Once the audit is complete, the audit is discarded and a new contract is dealt to replace it. If multiple audits are dealt in the same negotiation phase, the subsequent audits are simply discarded and replaced until there are no audits in play.

\subsection*{Acquisition}

A number of asset cards equal to the number of players are dealt face-up onto the table. Players take turns purchasing assets with unused budget starting with the first player and continuing clockwise. Once an asset is purchased, another asset should be dealt onto the table to replace it. This phase continues until there are no more players willing to purchase assets.

If there are already assets on the table from a previous round, additional assets should not be dealt.

\subsection*{Evaluation}

Each player loses satisfaction equal to the number of issues in their backlog. If any player is at full negative satisfaction, then the stability marker is moved a single space.

The total number of issues in all players' backlogs is totaled. The stability marker is moved a number of spaced equal to half this number (rounded down).

\section{Firing Assets}

At any time during their turn, a player may decide to fire one of their assets. Doing so puts the asset in the discard pile and returns any budget used to buy that asset to the player.

If at any time a player loses budget, they must fire enough assets to make up for that lost provided they don't have enough unspent budget to make up for the loss.

\section{Abandoning Contracts}

At any time during their turn, a player may choose to abandon a contract. Doing so allows the player to keep any budget associated with a contract while also ignoring any effects of the contract. The player should flip the contract face-down to signal that it is abandoned. Abandoned contracts cannot be un-abandoned.

\section{Running out of cards}

If at any time any of the decks are out of cards, the associated discard pile should be shuffled and used as the deck.

\end{document}
